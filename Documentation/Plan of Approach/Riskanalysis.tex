%FILE FOR RISK ANALYSIS
There are many risks and reasons why the project might fail or go bad, A change in organizational priorities is the most common reason. A change in project objectives is also common as poor communication and unclear risk definition.
	\begin{itemize}
		\setlength\itemsep{-0.3em}
		\item Unclear or shifting goals.
		\item lack of planning.
		\item lack of follow up.
		\item timing issues 
		\item Lack of risk management.
		\item Unsuitable tools.
		\item Too many unsuitable tools
	\end{itemize}

\noindent
For this reason, table \ref{tab:risk_analysis} has been created with ratings that describe the probability of the project experiencing delays due to a specific reason and the corresponding impact. By multiplying these values together, a risk score is generated for each risk. To have a high chance of successfully completing the project, it is not acceptable for this score to be bigger then or equal to ten.

\newpage
\begin{table}[!h]
	\centering
	\begin{tabular}{|>{\columncolor{gray}}c|c|c|c|c|c|}
		\cline{2-6}
		\rowcolor{gray}
		\multicolumn{1}{l|}{\cellcolor{white}}&Very low&Low		&Medium					&High					&Very high 				\\ \hline
		Very likely	&\cellcolor{yellow}5&\cellcolor{orange}10	&\cellcolor{red}15		&\cellcolor{red}20		&\cellcolor{red}25 		\\ \hline
		likely		&\cellcolor{green}4	&\cellcolor{yellow}8	&\cellcolor{orange}12	&\cellcolor{red}16		&\cellcolor{red}20 		\\ \hline
		possible	&\cellcolor{green}3	&\cellcolor{yellow}6	&\cellcolor{yellow}9	&\cellcolor{orange}12	&\cellcolor{red}15 		\\ \hline
		Unlikely	&\cellcolor{green}2	&\cellcolor{green}4		&\cellcolor{yellow}6	&\cellcolor{yellow}8	&\cellcolor{orange}10	\\ \hline
		Rare		&\cellcolor{green}1	&\cellcolor{green}2		&\cellcolor{green}3		&\cellcolor{green}4		&\cellcolor{yellow}5	\\ \hline
	\end{tabular}
	\caption{risk analysis index}
\end{table}

\renewcommand\tabularxcolumn[1]{m{#1}}% for vertical centering text in X column

\begin{table}[!h]
	\begin{tabularx}{\textwidth}{|c|X|X|c|c|c|} \hline
		\# 	& Risk 													& Action 																				& Probability 	& Impact 	& Score 	\\ \hline
		1 	& A group member failed a deadline			            & The group member will be addressed by the group regarding this and will reasonably do everything in their power to rectify the situation. If this does not happen, the respective member may be removed from the project.							   & 1 & 5			& \cellcolor{yellow}5	\\ \hline
		2 	& The work pace of the planning cannot be maintained. 	& In the planning it is taken into account that things can go wrong. Also buffer time will be included in the planning.																																		      & 3 & 2		 & \cellcolor{yellow}6	\\ \hline
		3 	& Messy file structure									& The version-controller looks after the structure of the folders						& 2	& 3			& \cellcolor{yellow}6	\\ \hline
		4 	& Behind planning for lack of knowledge or confusion	& Ask for help in time from fellow group members or teachers							& 2	& 3			& \cellcolor{yellow}6	\\ \hline
		5 	& There is insufficient communication in the group		& Weekly meetings where progress and tasks are discussed.								& 2	& 3			& \cellcolor{yellow}6	\\ \hline
		6 	& Clear results are missing								& The weekly meeting notes state clearly what activities have to be done the next week	& 1	& 3			& \cellcolor{green}3	\\ \hline
		7 	& Shortage in components								& Order parts early or look for available alternatives									& 2	& 4			& \cellcolor{yellow}8	\\ \hline
	\end{tabularx}
	\caption{Risk analysis}
	\label{tab:risk_analysis}
\end{table}

To reduce the risk for the project, the end goal will be achieved via an iterative design method. For the first iteration, a basic prototype will be made that will only be able to emulate two lithium-ion cells in series and cannot necessarily be controlled via software. Later iterations will build on this and add additional functionality. By expanding the design step by step in this way, parts can be tested at an early stage and adjusted if necessary.