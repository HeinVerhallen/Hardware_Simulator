The first iteration board takes as an input a voltage (12 V), which
is then scaled down to 5 V to power all the on-board electronics. 
There is a USB connector and a USB interface IC, which would ensure 
being able to program the microcontroller - an ATmega328PB. For The
first iteration no software is needed, but the option was added on the 
board (with a digital potentiometer, which is explained later).

In order to make the first iteration cycle quicker and focus on 
the modular design iteration, the team included a single cell 
only in the first PCB. The cell consists of:
\begin{enumerate}
    \item A power supply section, which provides 9 V and 5 V 
    (both isolated)
    \item A digital isolator - for communication for the second 
    iteration software
    \item A linear regulator - to provide the output voltage
    \item An op-amp and transistor circuit - for the current 
    sinking and sourcing
    \item Potentiometers (physical but also digital ones are on 
    the board for the second iteration) - to control the voltage
    and current
    \item Address jumpers - to distinguish each cell, when more 
    cells are added in the second iteration.
\end{enumerate}
