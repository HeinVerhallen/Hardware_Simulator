\IEEEPARstart
{F}{or} this project, the following non-functional requirements with their 
functional requirements have been specified. These requirements are set 
to ensure a successful iterative process, however, through the duration 
of the project some of them were changed. All the changes are explained
in the corresponding sections. The requirements go as follows:

\subsection{Non-functional requirements}
\begin{enumerate}
    \item[1.] Battery Pack emulator is able to emulate Li-ion characteristics.
    \item[2.] Battery pack emulator consists of two emulated Li-ion 
    cells(1st iteration).
    \item[3.] Battery pack emulator is able to emulate charging and discharging of
    Li-ion cells.
    \item[4.] The voltages of each emulated cell can be adjusted separately 
    via hardware (1st iteration).
    \item[5.] Each emulated cell has a voltage and current range that is 20\%
    larger than the safe operating area of a Li-ion cell.
    \item[6.] The voltages of each emulated cell can be adjusted separately 
    via software (2nd iteration).
    \item[7.] The battery pack emulator is able to emulate the temperature of 
    the battery pack.
    \item[8.] The battery pack emulator consists of 14 emulated Li-ion 
    cells (2nd iteration). 
    \item[9.] The load emulator is capable of emulating electronics and 
    motors (3rd iteration).
    \item[10.] The battery pack emulator and load emulator are safe. 
  \end{enumerate}

\subsection{Functional requirements}
Every functional requirement is linked to a non-functional requirement. 
\begin{enumerate}
    \item[1.1] The maximum voltage of the battery pack emulator is 80 V.
    \item[1.2] The battery pack emulator has a peak power of 13.44 kW 
    (see 10.1).
    \item[1.3] The battery pack emulator has a continuous power of 112 W 
    (see 10.1).
    \item[2.1] The first iteration of the battery pack emulator consists of 
    two emulated Li-ion cells.
    \item[3.1] The emulated cell is able to deliver current (see 5.2 and 5.5).
    \item[3.2] The emulated cell is able to sink current (see 5.4 and 5.5).
    \item[3.3] A linear regulator is used to generate the specified voltages 
    (see 5.1).
    \item[4.1] The cell voltages can be charged by a mechanical variable 
    resistor. 
    \item[5.1] The emulated cell has a voltage range of 2.48 V to 5.04 V.
    \item[5.2] The emulated cell has a peak sourcing current of 12 A.
    \item[5.3] The emulated cell has a continuous sourcing current of 100 mA.
    \item[5.4] The emulated cell has a peak sinking current of 12 A.
    \item[5.5] The emulated cell has a continuous sinking current of 100 mA.
    \item[5.6] The emulated cell has a peak power of 60 W.
    \item[5.7] The emulated cell has a continuous power of 0.5 W.
    \item[6.1] The battery emulator hardware uses a microcontroller. 
    \item[6.2] The user interface is made to adjust the voltages. 
    \item[6.3] The voltages are adjusted using a digital potentiometer. 
    \item[7.1] The temperature is emulated by using a digital potentiometer.
    \item[7.2] The emulated temperature range is -40° C to 80° C.
    \item[7.3] The emulated battery pack has an output which indicates the
    temperature relative to the output voltage.
    \item[8.1] 14 Li-ion cell emulators are wired in series. 
    \item[9.1] The load emulator is able to handle the battery pack voltage
    specified in 1.1.
    \item[9.2] The load emulator is able to handle the battery pack power
    specified in 1.2.
    \item[9.3] The load emulator can draw a current following a set-point 
    voltage which is generated by a microcontroller.
    \item[10.1] The temperature of the electronics should be less than 25° C.
    \item[10.2] The input of the battery pack emulator has reverse polarity
    protection.
\end{enumerate}