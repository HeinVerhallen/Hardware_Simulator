% \csvautotabular{TestPlan.csv}
% Table made with https://tableconvert.com/excel-to-latex
\begin{table*}[!ht]
    \centering
    \caption{Test Plan}
    \begin{tabular}{|p{0.3cm}|p{3cm}|p{7cm}|p{2.5cm}|p{2cm}|p{1cm}|}
        \hline
        "ID" & Requirement & Test method & Expected outcome & Measured outcome & Check \\ \hline
        1.1 & The maximum voltage of the 14 cell battery pack emulator is 80V & Attach Siglent SPD3303X-E power supply to input. Measure output voltage of the combined cells at maximum output voltage with a multimeter & 80V +- 10\% & N/A & ~ \\ \hline
        1.11 & The maximum voltage of the 4 cell battery pack emulator is 20V & Attach Siglent SPD3303X-E power supply to input. Measure output voltage of the combined cells at maximum output voltage with a multimeter & 20V +- 10\% & N/A & ~ \\ \hline
        1.2 & The battery pack emulator has a peak power of 13.44kW (see 10.1) & Attach Siglent SPD3303X-E power supply to input. Attach load to output and set output voltage of cell to maximum. Measure peak output voltage and current values with oscilloscope over time to calculate power.  & 13.44kW +- 10\% & N/A & ~ \\ \hline
        1.21 & The 4 cell battery pack emulator has a peak power of 240W (see 10.1) & Attach Siglent SPD3303X-E power supply to input. Attach load to output and set output voltage of cell to maximum. Measure peak output voltage and current values with oscilloscope over time to calculate power.  & 240W +- 10\% & N/A & ~ \\ \hline
        1.3 & The battery pack emulator has a continuous power of 112W (see 10.1) & Attach Siglent SPD3303X-E power supply to input. Attach load to output and set output voltage of cell to maximum. Measure output voltage and current with multimeter to calculate power.  & 112W +- 10\% & N/A & ~ \\ \hline
        5.1 & The emulated cell has a voltage range of 2.48 to 5.04 V & Attach Siglent SPD3303X-E power supply to input. Measure output voltage of each cell with multimeter at the minimum and maximum output voltages & 2.48V to 5.04V +-10\% & 2.5152V to 5.5104V & Pass \\ \hline
        5.2 & The emulated cell has a peak sourcing current of 12A & Attach Siglent SPD3303X-E power supply to input. Attach a load to the output and measure the current through the load & 12A +- 10\% & ~ & ~ \\ \hline
        5.3 & The emulated cell has a continuous sourcing current of 100mA & Attach Siglent SPD3303X-E power supply to input. Attach a load to the output and measure the current through the load & 100mA +- 10\% & 99.68 mA & Pass \\ \hline
        5.4 & The emulated cell has a peak sinking current of 12A & Take two Siglent SPD3303X-E power supplies in parallel and set the current to 6A and measure the current drawn from these supplies & 12A +- 10\% & ~ & ~ \\ \hline
        5.5 & The emulated cell has a continuous sinking current of 100mA & Take one Siglent SPD3303X-E power supply and set the current to 200mA and measure the current drawn from the supply & 100mA +- 10\% & ~ & ~ \\ \hline
        5.6 & The emulated cell has a peak power of 60W & Attach Siglent SPD3303X-E power supply to input. Attach load to output and set output voltage of cell to maximum. Measure peak output voltage and current values with oscilloscope over time to calculate power.  & 60W +- 10\% & ~ & ~ \\ \hline
        5.7 & The emulated cell has a continuous power of 0.5W & Attach Siglent SPD3303X-E power supply to input. Attach load to output and set output voltage of cell to maximum. Measure output voltage and current with multimeter to calculate power.  & 0.5W +- 10\% & 4.99 mW & \hl{Fail} \\ \hline
        6.2 & A user interface is made to adjust the voltages & Set-up the user device and attach it to the PCB. Test for specific voltages, voltages on the boundaries, and if the voltages change in real time. Measure the output with a multimeter  & Output voltage responds to changes in the user interface & Output voltage responds to changes in the user interface & Pass \\ \hline
        6.3 & The voltages are adjusted using a digital potentiometer & Set-up the user device and attach it to the PCB. Vary the resistance of the potentiometer to achieve specific voltages and measure the output voltage of the PCB using a multimeter.   & Output voltage responds to changes in the digital potentiometer  & Output voltage responds to changes in the digital potentiometer & Pass \\ \hline
        7.1 & The temperature is emulated by using a digital potentiometer & Set-up the device and decide on reference temperatures. Set the digital potentiometer to the desired emulated temperature and measure.  & The emulated temperature matches the temperature from the potentiometer & N/A & ~ \\ \hline
        7.2 & The emulated temperature range is -40 to 80 C & Set-up the device. Set the digital potentiometer to emulate the desired temperature boundaries and measure.  & The emulated temperature range matches the desired range for the device  & N/A & ~ \\ \hline
        7.3 & The emulated battery pack has an output which indicates the temperature relative to the output voltage & ~ & ~ & N/A & ~ \\ \hline
        10.1 & The temperature of the electronics should be less than 25 C & *Run the device for an hour and then take the temperature of the electronics & Temperature is under 25 C & Temperature is under 25 C & Pass \\ \hline
        10.2 & The input of the battery pack emulator has reverse polarity protection & Wire the input with inverse polarity  & Nothing blows up & Nothing blows up & Pass \\ \hline
        10.3 & The input of the battery pack emulator and load emulator have over current protection & Increase the input current above the maximum value  & Nothing blows up  & Nothing blows up & nothing blows up \\ \hline
    \end{tabular}
    \label{Test Plan}
\end{table*}