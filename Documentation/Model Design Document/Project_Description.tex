\IEEEPARstart
{T}{he} goal of this project is to design, build, and test a battery pack emulator and (optionally)
a load emulator. The positive outcome of this development ensures a safe way of testing the BMS.


The first iteration of the design includes a Li-ion battery cell emulator, capable of charging and discharging small currents (100 mA) and
the voltage of the emulated cell is adjusted by a mechanical variable resistor. This iteration has no software and it is needed only for 
testing the current sourcing and sinking. This design is completed on a single PCB.

The second design iteration is made modular - consisting of a model board PCB and cell PCBs, and software is implemented to control the 
voltages of each emulated cell. The model board PCB has the microprocessor and is able to connect to and control up to 14 cell PCBs. Each
cell PCB includes all the circuitry needed to source and sink current up to 12 A, control the voltage and emulate the temperature via software.
Each emulated cell has a voltage and current range that is 20\% larger than the safe operating area of a Li-ion cell.



% Because the project team members are not yet experts in the field of designing battery pack
% emulators for BMS testing, it is chosen to design the battery pack emulator in iterations. In
% the first design iteration a battery pack emulator circuit will be designed which emulates the
% characteristics of a Li-ion battery pack. This battery pack emulator consists of two Li-ion cell
% emulators which are capable of charging and discharging. The voltages of each emulated cell are
% adjusted by a mechanical variable resistor. Each emulated cell has a voltage and current range
% that is 20\% larger than the safe operating area of a Li-ion cell.

% In the second design iteration software will be added to the battery emulator circuit to
% control the voltages of each emulated cell via software. Also the temperature will be emulated
% in the battery pack emulator. The third design iteration consists of expanding the amount of
% emulated cells in the emulated battery pack to 14 cells. In the fourth design iteration a load
% emulator will be designed to test the BMS under different load conditions. This load emulator
% is capable of emulating electronics and motors.

% At the end of the project it is expected that a battery pack emulator and optionally a load
% emulator is delivered. And if the planning allows it, a research report on existing battery and
% load emulators and the requirements for future battery and load emulators is delivered.
% % \hfill mds
 
% \hfill August 26, 2015

% \subsection{Subsection Heading Here}
% Subsection text here.

% % needed in second column of first page if using \IEEEpubid
% %\IEEEpubidadjcol

% \subsubsection{Subsubsection Heading Here}
% Subsubsection text here.