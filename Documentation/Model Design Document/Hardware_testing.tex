\subsubsection{Hardware testing}
During the testing phase of the second iteration of the cell board, the digital isolator was compromised due to 6.5 V being present on the 5 V rail. This was caused by a mistake in the PCB that swapped the collector and the base pins of the linear regulator series pass transistors, providing a current path through the collector base diode of the PNP transistors. Hereby, the output voltage minus one diode drop was present on the set pins of the LT3080-1. Due to the build in ESD protection diodes of the digital potentiometers (MCP4651T-103E/ST) this higher voltage than its supply voltage of 5 V facilitated a current path through the positive ESD protection diode to the 5 V rail, pulling it to 6.5 V. By this the digital potentiometer heated up a bit but furthermore does not seem to be compromised since it can handle a supply voltage of up to 7 V. The digital I2C isolator (ISO1640BDR) was compromised by this increase in supply voltage since it works only up to 5.5 V and can handle up to 6 V. It is beleaved that the OpAmp (MCP6001UT-I/OT) of the current sink was also damaged by this overvoltage so it was also replaced.

A simple fix was made by cutting the legs of the base and collector of the series pass transistors and swapping those connections with some wires. After this, the digital isolator from the first iteration PCBA was used to replace the broken one on the second iteration, which solved the problem.

Furthermore, due to the last minute implementation of the new power linear regulator based on the LT3080-1 a mismatch between the simulation and the schematic occurred. Therefore, the linear regulator needed a 500 k$\tcohm$ potentiometer which was not present on the PCBA of the second iteration. To be able to proceed with the testing and get as much knowledge from this iteration as possible, the wiper trace of the digital potentiometer connected to the set pints of the LT3080-1 IC's was cut with a scalpel and first connected to a 500 k$\tcohm$ trimmer to test the functionality of the circuit.

After verifying that the circuit works, a new I2C digital potentiometer was ordered with a higher resistance, and soldered on the cell board with some wires.

After these updates, the cell board is working as intended thus far.