\subsubsection{Software testing}
The first software tests were to test the communication between the microcontroller and the digital potentiometer. The digital potentiometer was connected to the microcontroller using the I2C bus. The resistance was set via software, and measured to confirm the correct working of the potentiometer and the cell class.

Before the software could be tested, the microcontroller (Atmega328p) had to be bootloaded, however, while designing the PCB the bootloading pins were left disconnected. This meant that the microcontroller had to be bootloaded using the ICSP pins. The ICSP pins were connected to the microcontroller using jumper wires. The microcontroller was bootloaded using the Arduino IDE and an Arduino Uno as an ISP. After the succesfull bootloading of the microcontroller the USB to UART converter (ft230) chip had to be tested to upload code to the microcontroller. 

It is in this step that the next problem arose. The USB to UART converter chip was not detected as a USB device by Windows. After some research it was suspected that the USB data lines did not have the proper impedance matching series resistors, which could have been the cause for the problem but could not have been verified since no data was measured at the data lines of the USB-C connector. It was also discovered that the commonly used 100 nF series capacitor to let the ft230 reset the microcontroller was not implemented. Due to these problems and considering the timing and planning, the decision was made to not focus on a redesign of the model board, and continuing the tests with a single cell. An Arduino Uno was used as the controller for the single cell.

These tests concluded that the resistance value for the voltage source was too low. It was at 10k$\Omega$, which gives a range of 200mV. This was too low for the cell to be able to reach the desired voltage. The potentiometer was replaced with a 200k$\Omega$ potentiometer, which gives a range of 3V. This was more than enough for the cell to reach the voltage range of 2.48-5.04V.